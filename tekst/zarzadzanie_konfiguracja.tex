\chapter{Zarządzanie konfiguracją}
W rozdziale tym przedstawię różne metody zarządzania konfiguracją serwerów. Postaram się opisać poglądowo różne metody, jak również przedstawić zalety i wady poszczególnych z nim.
\section{Ręczna konfiguracja każdego serwera za pomocą SSH}
\subsection{Opis}
Ręczna konfiguracja serwerów stososowana jest głównie tam, gdzie administrator ma pod swoją opieką jeden bądź kilka serwerów. W takim przypadku zmiana konfiguracji na serwerze jest prosta i nie zajmuje dużej ilości czasu.\\
Konfiguracja taka nie wymaga od administratora żadnej wiedzy wykraczającej poza obszar konfigurowanego systemu oraz usług, a wprowadzane zmiany widoczne są od razu po wprowadzeniu.
Ten sposób konfiguracji spotykany jest czasem w większych systemach informatycznych.
Dzieje się tak zwykle w jednostkach szybko rozwijających się, gdzie nastąpił szybki wzrost liczby serwerów i nie opracowano jeszcze metoda automatyzacji konfiguracji.

Do konfiguracji ręcznej nie potrzeba żadnego dodatkowego oprogramowania ani po stronie maszyn konfigurowanych, ani maszyny z której następuje konfiguracja.
Na maszynie z której następuje konfiguracja musi być dostępny klient SSH, który jest instalowany domyślnie we wszystkich dystrybucjach systemów GNU/Linux, a na maszynach konfigurowanych musi być zainstalowany i uruchomiony serwer SSH - jest on domyślnie zainstalowany w większości dystrybucji serwerowych GNU/Linux i w części dystrybucji przeznaczonych na komputery domowe.

Wadą takiej metody jest również sytuacja, w której tylko jedna osoba, bądź mała grupa osób, zna konfigurację poszczególnych serwerów oraz usług.
W przypadku opuszczenia przez daną osobę zespołu, pozostali członkowie muszą, analizując pliki konfiguracyjne, zrozumieć zamysł osoby to tworzącej.\\
Kolejną wadą, jest brak możliwości powielenia konfiguracji.
W przypadku gdy zaistnieje potrzeba skonfigurowania bliźniaczego serwera, jako serwera zapasowego, należy każdą usługę skonfigurować od nowa na wzór serwera pierwotnego. Również wprowadzane zmiany należy uwzględniać na wszystkich serwerach.
Może to w prosty sposób prowadzić do błędów i rozbierzności konfiguracji.
\subsection{Zalety i wady}
Zalety:
\begin{itemize}
\item prostota
\item używanie tylko domyślnych komponentów systemu
\item szybkość wprowadzanych zmian
\item informacja zwrotna czy usługa została uruchomiona poprawnie
\end{itemize}
Wady:
\begin{itemize}
\item \textbf{brak skalowalności}
\item różnice między poszczególnymi serwerami
\item trudność powielania
\item wiedza o konfiguracji zależna od jednego pracownika
\end{itemize}
\subsection{Przykład}
\lstinputlisting{lst/conf_ssh.sh}
\subsection{CSSH}
Istnieje narzędzie CSSH (\textit{Cluster SSH}) które wychodzi na przeciw osobą chcącym konfigurować kilka serwerów jednocześnie poprzez SSH.
Narzędzie to potrafi otworzyć wiele sesji SSH równolegle - każda sesja w osobnym terminalu.
Głównym iterface-em programu, jest małe okno wejścia, które przechwytując wpisywany do niego tekst, przesyła go do wszystkich otwartych sesji.\\
Zmniejsza to prawdopodobnieństwo rozbierzności w konfiguracji, jak również przyśpiesza proces, ponieważ tekst jest wpisywany do wszystkich sesji jednocześnie i nie ma potrzeby wielokrotnego wpisywania tej samej konfiguracji na wielu maszynach.\\
Aplikacja umożliwia również przełączenie się w dowolnej chwili na konkretny terminal i interakcję tylko z jednym serwerem, np: w celu zdiagnozowania problemu występującego tylko na tej jednej maszynie.
\section{Fabric}
\subsection{Opis}
Jest aplikacją napisaną w języku Python, służącą głównie do wykonywania poleceń powłoki na zdalnym serwerze. Aplikacja pozwala na zdefiniowanie kolejności w jakiej mają zostać poszczególne polecenia, jak również udostępnia kilka funkcji sprawdzających, np: czy plik istnieje, bądź kopiowanie plików na lub z serwera.\\

\subsection{Zalety i wady}
\subsection{Przykład}
