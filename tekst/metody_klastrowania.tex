\chapter{Metody klastrowania}
W rozdziale tym przedstawię podstawowe metody wykorzystywane przy tworzeniu klastrów pod aplikacje internetowe.
\section{DNS round robin}
\subsection{Czym jest DNS}
DNS (ang {\textit{Domain Name System}) jest to system nazw domenowych. Usługa której najważniejszą funkcją jest przyporządkowanie nazwom domenowym (czytelnym dla człowieka) adresów IP.
Oznacza to, że system taki, jest w stanie zamienić nazwę \texttt{www.ftj.agh.edu.pl} na adres \texttt{149.156.110.3}.
Funkcjonalność taka w znacznym stopniu ułatwia korzystanie z sieci Internet, ponieważ przeciętnemu człowiekowi jest prościej zapamiętać mnemonik \texttt{www.ftj.agh.edu.pl} bądź \texttt{www.duckduckgo.com} niż ciągi czterech liczb.
Do zamiany nazwy domenowej na adres IP służą rekordy \texttt{A} oraz \texttt{AAAA}, wykorzystywane odpowiednio do adresów IPv4 oraz IPv6.
Przykład zastosowania rekordu \texttt{A} przedstawiam na poniższym listingu:
\lstinputlisting{lst/rr_arecord.lst}
w powyżej konfiguracji widzimy, że odpytując server DNS o wartość \texttt{nazwa4.domenowa.pl} otrzymamy informację, że dana nazwa wskazuje na adres \texttt{1.2.3.4}.\\
Innymi, często spotykanymi rekordami są rekordy \texttt{CNAME}, \texttt{MX} oraz \texttt{TXT}.
\begin{description}
\item[CNAME] jest to rekord będący wskaźnikiem. Dla przykładu:
\lstinputlisting{lst/rr_cname.lst}
widzimy, że w powyższym przykładzie \texttt{nazwa.domenowa.pl} wskazuje na 1.2.3.4.
Chcąc aby, \texttt{www.nazwa.domenowa.pl} również wskazywała w to samo miejsce, moglibyśmy również zdefiniować ją jako rekord \texttt{A} z takim samym adresem.
Jednak, w przypadku migracji serwera z adresu \texttt{1.2.3.4} na \texttt{1.2.3.5}, należałoby zmieniać ten adres w obu rekordach.
Zastosowanie rekordu \texttt{CNAME}, pozwala powiedzieć "nazwa \texttt{www.nazwa.domenowa.pl} wskazuje w to samo miejsce, w które wskazuje \texttt{nazwa.domenowa.pl}".
Prowadzi to do zmniejszenia ryzyka pomyłki przy wpisywaniu adresów IP, jak również zmniejsza liczbę miejsc w których należy zmienić adresację w przypadku migracji serwera.
\item[MX] rekord wskazujący na adres serwera pocztowego obsługującego daną domenę.
Strefa może zawierać kilka rekordów \texttt{MX} w celu dystrybucji ruchu na kilka serwerów pocztowych.
\lstinputlisting{lst/rr_mxrecord.lst}
Widzimy, że wpis definiujący rekord \texttt{MX} nie posiada nazwy. Zwykle rekord ten definiowany jest na początku definicji strefy, dlatego pominięcie nazwy powoduje, ze rekord ten odnosi się do nazwy tej strefy.
Jest to kolejna rzecz które uogólnia konfigurację i ułatwia migrowanie.
Wartość \texttt{5} oznacza poziom preferencji danego serwera. Mając zdefiniowanych kilka rekordów \texttt{MX}, poczta jest dystrybuowana przy pomocy algorytmu ważonego round robin.
\item[TXT] rekord który zgodnie z założeniami DNS miał zawierać dane tekstowe czytelne dla człowieka.
Obecnie rzadko zawiera dane dla użytkowników. Wykorzystywany jest głównie do konfiguracji SPF, co wykracza poza tematykę tej pracy.
\end{description}
\subsection{Opis metody}
Metoda ta polega na odpowiednim skonfigurowaniu strefy na serwerze DNS, w taki sposób, aby pod jedna nazwa rozwiązywała się na kilka adresów IP.
W efekcie, gdy serwer otrzyma zapytanie o daną nazwę domenową, zostanie mu zwrócona pula adresów zamiast jednego.
Aplikacja która otrzyma listę adresów IP, powinna połączyć się na losowy z nich.
Niestety nigdy nie ma pewności, że aplikacja posiada zaimplementowaną obsługę wielu adresów zwracanych przez serwer DNS, dlatego serwer wprowadza zabezpieczenie przed takim zachowaniem, a mianowicie tytułowy algorytm \textit{round robin}, który zwraca adresy IP, jednak za każdym razem ich permutację.\\
Dla przykładu, poniżej zamieszczone trzy zapytania wykonane po sobie.
\lstinputlisting{lst/rr_dig.shell}
Widzimy, że przy każdym zapytaniu, jako pierwszy adres zwracany jest kolejny adres z puli. Zapewnia to prawidłowe balansowanie ruchu, nawet przy aplikacjach nie potrafiących obsłużyć wielu adresów i łączących się na pierwszy otrzymany.

Metoda ta jest metodą wysokiej wydajności, ponieważ pozwala w sposób niewidoczny dla użytkownika rozdzielić ruch na kilka serwerów, a tym samym rozłożyć obciążenie, to skutkować będzie szybszą odpowiedzią klientowi na zapytanie.
Metoda ta nie zapewnia natywnie wykrywania niedostępności któregoś z serwerów, dlatego nie może służyć bezpośrednio jako metoda wysokiej dostępności.
Pośrednio występuje tutaj jednak mechanizm broniący przed niedostępnością któregoś z serwerów. W przypadku gdy aplikacja próbować się będzie połączyć z losowym adresem z puli, a połączenie nie będzie mogło być nawiązane, osiągnięty zostanie limit czasu połączenia (tzw. \textit{timeout}. W takiej sytuacji, dobrze napisana aplikacja, będzie próbować połączyć się na kolejny adres z puli, w nadziei, że będzie on dostępny.
W takiej sytuacji, połączenie zostanie nawiązane i klient otrzyma odpowiedź, jednak do czasu generowania odpowiedzi, trzeba doliczyć czas potrzebny na osiągnięcie \textit{timeout-u}. Może on wynieść od kilku, do kilkudziesięciu sekund.

Metoda ta jest również zależna od działania serwera DNS.
Najprostszym sposobem ochrony przed awarią tego systemu dystrybucji ruchu jest skonfigurowanie \textit{Secondary DNS}. To jednak wykracza poza tematykę tej pracy.
\subsection{Konfiguracja}
Konfiguracja DNS round robin jest stosunkowo prosta.
W konfiguracji strefy, należy umieścić wpis z wieloma rekordami \textit{A} dla jednej nazwy.
Przykład takiej strefy zamieszczony został poniżej
\lstinputlisting{lst/rr_zone.zone}
W powyższym przykładzie, dla nazwy \texttt{rr.mgr.fabrykowski.pl} zostały zdefiniowane trzy adresy IP.
\section{Nginx}
\subsection{Czym jest nginx}
Nginx jest serwerem proxy oraz serwerem treści statycznych.
Wykorzystywany jest zwykle w połączeniu z serwerem Apache który serwuje treści PHP, podczas gdy sam dostarcza pliki statyczne (JavaScript, CSS, JPEG itp).
Drugim często wykorzystywanym modelem wykorzystania Nginx-a jest serwowanie treści statycznych oraz wykonywanie \textit{fastcgi pass}
\paragraph{Fastcgi pass} \hspace{0pt} \\
Moduł ten pozwala na komunikacje z procesami FastCGI.
Wykorzystanie FastCGI daje dużą niezależność w technologi opracowania aplikacji, która może zostać wykonana w PHP, Pythonie bądź Rubym.
Istnieje również możliwość zmiany wersji aplikacji, bądź technologii jej wykonania bez zmian w konfiguracji serwera, jeżeli aplikacja udostępnia to samo api FastCGI.

Do obsługi języka PHP zostanie wykorzystany php-fpm (\textit{PHP FastCGI Process Manager}.
Jest to alternatywna implementacja PHP FastCGI.
Pozwala ona na większą kontrolę w zakresie pul procesów - ich liczby oraz sposobu uruchamiania, jak również dowolność w kwestiach sieciowych - adres oraz port do nasłuchiwania.

Porównanie testów wydajności PHP-fpm oraz mod_php do Apache jak również szybkość serwowania treści statycznych zostanie przedstawione w późniejszych rozdziałach.
\subsection{Opis metody}
Metoda klastrowania przy pomocy Nginx-a polega na zdefiniowaniu sekcji \texttt{upstream}.
Pozwala to na skonfigurowanie puli adresów do których będą przekazywane zapytania.
Aby dodać serwer do puli, należy podać jego adres IP bądź nazwę domenową oraz port.

Zapytania do serwerów wykonujących (\textit{workerów}) rozdzielane są równomiernie pomiędzy wszystkie serwery w puli.\\
Pozwala to na obsługiwanie zapytań na wielu maszynach, dlatego metoda ta pozwala na tworzenie klastrów \textbf{wysokiej wydajności}.

Ponadto, zaimplementowany jest również mechanizm sprawdzający stan poszczególnych serwerów w puli i w przypadku wykrycia awarii, oznaczany jest on jaki \textit{failure} i zapytanie nie są do niego kierowane.\\
Jest to zachowanie typowe dla klastrów \textbf{wysokiej dostępności}

Istnieje możliwość modyfikacji domyślnego algorytmu używanego przez Nginx-a.
\begin{itemize}
	\item zmiana sposobu dystrybucji zapytań dodając opcjonalny parametr \texttt{weight} mówiący o wadze danego węzła. Dla przykładu, w poniższej konfiguracji:
	\lstinputlisting{lst/nx_weight.conf}
	na każde 6 zapytań do \texttt{pula1}, 5 zostanie przekazanych do \texttt{server1} a jedno do \texttt{server2}.
	Opcja ta wykorzystywana jest głównie tam, gdzie poszczególne serwery różnią się parametrami bądź obciążeniem nie wynikającym z obsługiwania tej puli.
	\item zmiana sposobu określania serwera jako niedostępnego.
	Służą do tego parametry \texttt{max_fails}, \texttt{fail_timeout} oraz \texttt{slow_start}.\\
	\begin{description}
	\item[\texttt{max_fails}] określa liczbę nieudanych prób komunikacji z serwerem w czasie \texttt{fail_timeout} nim serwer zostanie oznaczony jako niedostępny.
	Domyślna wartość tego parametru wynosi 1, natomiast wartość 0 wyłącza oznaczanie serwerów jako niedostępne.
	\item[\texttt{fail_timeout}] określa czas w jakim musi nastąpić \texttt{max_fails} nim serwer zostanie uznany za niedostępny.
	Określa również interwał czasowy co który będzie sprawdzana dostępność serwera.
	Wartość domyślna dla tego parametru wynosi 10 sekund.
	\item[\texttt{slow_start}] określa czas w jakim będzie zwiększana wartość \texttt{weight} od zera do docelowej po przejściu serwera ze stanu niedostępnego do stanu dostępnego.
	Wartość domyślna wynosi 0, co oznacza wyłączone płynne włączanie serwera do puli.
	\end{description}

\end{itemize}
\subsection{Konfiguracja}
\section{Haproxy}
\subsection{Czym jest haproxy}
\subsection{Konfiguracja}
