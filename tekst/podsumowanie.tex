\addcontentsline{toc}{chapter}{Podsumowanie}
\chapter*{Podsumowanie}
Niniejsza praca wykazała potrzebę stosowania klastrów WWW.
Przy obecnym rozwoju internetu, ilość zapotrzebowania na dane jest daleko wykraczająca poza możliwości pojedynczych komputerów.
Ponadto, niemożność dostarczenia klientowi żądanych danych jest równoznaczne z ponoszonymi przez firmę stratami finansowymi oraz wizerunkowymi.

Obecnie każdy popularny serwis internetowy wykorzystuje różnego rodzaju klastry.
Poziom zaawansowanie konfiguracji w przypadku dużych firm wykracza daleko poza możliwości opisywanego systemu.

Ponadto, w niniejszej pracy zostały opisane tylko metody \textit{software}owe, a zostały pominięte metody \textit{hardware}owe.
Wynika to z braku dostępu do profesjonalnego sprzętu.\\
Dodatkowo, z racji objętości pracy, pominiętę zostały niektóre, bardziej zaawansowane metody wykorzystywane przez duże korporacjie, np: zastosowanie modułu \texttt{GeoIP}, pozwalającego na serwowanie różnej treści klientowi w zależności od jego położenia geograficznego.
Wykorzystywane jest to np: przy zwracaniu różnych adresów IP dla danej domeny.
Klientowi zwraca się adres serwera w \textit{datacenter} znajdującego się najbliżej, w celu zmiejszenia czasu odpowiedzi.\\
Nie został również poruszony temat tworzenia własnych modułów do \texttt{nginx}a pozwalających dostosować sposób dystrybucji ruchu do konkretnych warunków panujących w firmie.\\
Nie został również przetestowany serwer \texttt{lighthttp}.
Wynika to z faktu dość specyficznej charakterystyki tego serwera, sprawiającej że potrafi w szybki sposób serwować pliki statyczne, lecz posiada mniejsze możliwości konfiguracyjne niż \texttt{apache} lub \texttt{nginx}.\\
Innym ważnym zaganieniem nieporuszonym w niniejszej pracy, jest użycie \texttt{keepalived} bądź innego narzędzia wspomagającego wykrywanie i wypinanie nieaktywnych węzłów z klastra \texttt{LVS}.\\
Nieopisany również został mechanizm \textit{floating IP}, powodujący na dynamiczne przepinanie publiczego adresu IP na zapasowy (\textit{host swap}) węzeł - np: zapasowy \texttt{director}.

Jednak opisany system jest w stanie wystarzyć dla małej lub średniej strony.
Wykorzystuje on najczęściej używane metody klastrowania i daje duże możliwości profilowania konfiguracji na potrzeby konkretnego rozwiązania.

Trzeba jednak pamiętać, że nawet najlepszy klaster wysokiej wydajności nie zastąpi optymalizacji aplikacji działającej 
