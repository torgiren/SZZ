\addcontentsline{toc}{chapter}{Wstep}
\chapter*{Wstęp}
Niniejsza praca opisuje aplikację ułatwiającą administratorowi tworzenie oraz administrację klasterm dla aplikacji WWW.\\
Została ona podzielona na pięć rozdziałów tworzących trzy grupy.\\
Pierwsza grupa skupia się na rozważaniach teoretycznych.
W jej skład wchodzą rozdziały:
\begin{description}
\item{"Wstęp do klastrowania"}\\
Opisuje on powody z których zaczęto stosować klastry WWW.
Przybliża podział klastrów oraz cechy każdego rodzaju.
Dodatkowo zarysowywuje problem wynikający z trudności zarządzania konfiguracją na klastrach.
\item{"Metody klastrowania"}
Rozdział ten opisuje najbardziej znane metody służące klastrowaniu aplikacji.
Opisuje on metody takie jak \textit{DNS round robin}, \textit{Nginx upstream}, \textit{Haproxy}, \textit{LSV}.
Każda z powyższych metod zostałą krótko opisana, sposób jej działania oraz sposób konfiguracji.
\item{"Zarządzanie konfiguracja"}\\
Ten rozdział przedstawia problem konsystentnej konfiguracji dużej ilości serwerów.
Opisuje on podstawowy, ręczny sposób konfiguracji maszyn.
Jego zalety i wady, a następnie kolejne metody usprawniające i eliminujące wady ręcznej konfiguracji.
Zostały opisane aplikacje takie jak: \textit{Fabric}, \textit{Puppet}, \textit{CFEngine}, \textit{Ansible}, ze szczególnym naciskiem na ostatni.
\end{description}

W drugiej części pracy, znajduje się jeden rozdział, opisujący praktyczne testy wydajnościowe opisywanych wcześniej metod klastrowania.
Skupia się on zarówno na testowaniu możliwości zwiększania wydajności jak i zapewnianiu wysokiej dostępności.

Trzecia część opisuje tytułowy \textit{System zautomatyzowanego zarządzania konfiguracją farmy serwerów aplikacji WWW}.
Opisuje on powody wyboru konkretnych technologii, opis ich konfiguracji oraz sposobu współdziałania.
