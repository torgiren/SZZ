\documentclass[a4paper,12pt,oneside]{book}
\usepackage[utf8]{inputenc}
\usepackage[T1]{polski}
\usepackage{polski}
\usepackage{color}
\usepackage{helvet}
\usepackage{graphicx}
\usepackage{caption}
\usepackage{subcaption}
\usepackage{geometry}
\usepackage{titlesec}
\usepackage{indentfirst}
\usepackage{verbatim}
\usepackage{moreverb}
\usepackage{nameref}
\usepackage{url}
\usepackage{xcolor}
\usepackage[font=footnotesize,labelfont=bf]{caption}
\usepackage{multirow}
\usepackage{booktabs}
\usepackage{float}
\geometry{hmargin={2cm, 2cm}, height=10.0in}
\assignpagestyle{\chapter}{empty}
\newcommand{\param}[1]{\textit{\textless #1\textgreater}}

\DeclareUnicodeCharacter{00A0}{~}

%konfiguracja listingow
\usepackage{listings}
\DeclareCaptionFont{white}{\color{white}}
\DeclareCaptionFormat{listing}{\colorbox{gray}{\parbox{\textwidth}{#1#2#3}}}
\captionsetup[lstlisting]{format=listing,labelfont=white,textfont=white}
\lstset{basicstyle=\footnotesize\ttfamily}
\newcommand{\wykresy}[3] {
	\begin{figure}[H]
		\centering
		\begin{subfigure}{0.48\textwidth}
			\includegraphics[width=\textwidth]{testy/#1_1.png}
			\caption{1 równoległe zapytanie}
		\end{subfigure}
		\begin{subfigure}{0.48\textwidth}
			\includegraphics[width=\textwidth]{testy/#1_2.png}
			\caption{2 równoległe zapytania}
		\end{subfigure}
	\end{figure}

	\begin{figure}[H]
		\ContinuedFloat
		\begin{subfigure}{0.48\textwidth}
			\includegraphics[width=\textwidth]{testy/#1_4.png}
			\caption{4 równoległe zapytania}
		\end{subfigure}
		\begin{subfigure}{0.48\textwidth}
			\includegraphics[width=\textwidth]{testy/#1_8.png}
			\caption{8 równoległych zapytań}
		\end{subfigure}
	\end{figure}

	\begin{figure}[H]
		\ContinuedFloat
		\begin{subfigure}{0.48\textwidth}
			\includegraphics[width=\textwidth]{testy/#1_16.png}
			\caption{16 równoległych zapytań}
		\end{subfigure}
		\begin{subfigure}{0.48\textwidth}
			\includegraphics[width=\textwidth]{testy/#1_32.png}
			\caption{32 równoległe zapytania}
		\end{subfigure}
	\end{figure}
	
	\begin{figure}[H]
		\ContinuedFloat
		\begin{subfigure}{0.48\textwidth}
			\includegraphics[width=\textwidth]{testy/#1_64.png}
			\caption{64 równoległe zapytania}
		\end{subfigure}
		\begin{subfigure}{0.48\textwidth}
			\includegraphics[width=\textwidth]{testy/#1_128.png}
			\caption{128 równoległe zapytania}
		\end{subfigure}
	\end{figure}

	\begin{figure}[H]
		\ContinuedFloat
		\centering
		\begin{subfigure}{0.48\textwidth}
			\includegraphics[width=\textwidth]{testy/#1_256.png}
			\caption{256 równoległych zapytań}
		\end{subfigure}
		\caption{#2}\label{fig:#3}
	\end{figure}
}


\title{System zautomatyzowanego zarządzania konfiguracją farmy serwerów aplikacji WWW}
\author{Marcin TORGiren Fabrykowski}
\setcounter{secnumdepth}{4}
\setcounter{tocdepth}{4}
\begin{document}
\nocite{*}
%\bibliographystyle{plain}
\bibliographystyle{amsplain}
\thispagestyle{empty}
\includegraphics[height=37.5mm]{agh_nzw_a_pl_1w_wbr_cmyk.eps}\\
\rule{30mm}{0pt}
{\large\textsf{Wydział Fizyki i Informatyki Stosowanej}}\\
\rule{\textwidth}{3pt}\\
\rule[2ex]
{\textwidth}{1pt}\\
\vspace{7ex}
\begin{center}
{\bf\LARGE\textsf{Praca magisterska}}\\
\vspace{13ex}
% --------------------------- IMIE I NAZWISKO -------------------------------
{\bf\Large\textsf{Marcin Fabrykowski}}\\
\vspace{3ex}
{\sf \small kierunek studiów:} {\bf\small\textsf{informatyka stosowana}}\\
{\sf \small specjalność:} {\bf\small\textsf{grafika komputerowa i przetwarzanie obrazów}}\\
\vspace{15ex}
%% ------------------------ TYTUL PRACY --------------------------------------
{\bf\huge\textsf{System zautomatyzowanego zarządzania farmą serwerów aplikacji WWW}}\\
\vspace{14ex}
%% ------------------------ OPIEKUN PRACY ------------------------------------
{\sf \Large Opiekun:} {\bf\Large\textsf{dr inż. Piotr Gronek}}\\
\vspace{22ex}
\textsf{\bf\large\textsf{Kraków, sierpień 2015}}
\end{center}
%% =====  STRONA TYTUŁOWA PRACY INŻYNIERSKIEJ  ====

\newpage

%% =====  TYŁ STRONY TYTUŁOWEJ PRACY INŻYNIERSKIEJ  ====
{\sf Oświadczam, świadomy(-a) odpowiedzialności karnej za poświadczenie nieprawdy, że niniejszą pracę dyplomową wykonałem(-am) osobiście i samodzielnie i nie korzystałem(-am) ze źródeł innych niż wymienione w pracy.}

\vspace{14ex}

\begin{center}
\begin{tabular}{lr}
~~~~~~~~~~~~~~~~~~~~~~~~~~~~~~~~~~~~~~~~~~~~~~~~~~~~~~~~~~~~~~~~~ &
................................................................. \\
~ & {\sf (czytelny podpis)} \\
\end{tabular}
\end{center}

\newpage
\rightline{Kraków, 14 września 2015}
\begin{center}
{\bf Tematyka pracy magisterskiej i praktyki dyplomowej
Marcina Fabrykowskiego.
studenta V roku studiów kierunku informatyka stosowana, specjalności grafika komputerowa i przetwarzanie obarzów.}\\

\end{center}

Temat pracy magisterskiej:
{\bf System zautomatyzowanego zarządzania farmą serwerów aplikacji WWW}\\

\begin{tabular}{rl}

Opiekun pracy:                  & dr inż. Piotr Gronek\\
Recenzenci pracy:               & dr inż. Antoni Dydejczyk\\
Miejsce praktyki dyplomowej:    & Inittec Sp. z o.o., Kraków\\
\end{tabular}

\begin{center}
{\bf Program pracy magisterskiej i praktyki dyplomowej}
\end{center}

\begin{enumerate}
\item Omówienie realizacji pracy magisterskiej z opiekunem.
\item Praktyka dyplomowa:
\begin{itemize}
\item zapoznanie się z rozwiązaniami stosowanymi w dużych środowiskach informatycznych
\item administracja istniejącą infrastrukturą klastrową
\end{itemize}
\item Zapoznanie się z innymi metodami tworzenia klastrów oraz zarządania konfiguracją
\item Testowanie poznanych rozwiązań
\item Ręczne stworzenie klastra WWW oraz sprawdzenie jego działania
\item Opracowanie metody automatycznego konfigurowania maszyn
\item Opracowanie redakcyjne pracy.
\end{enumerate}


\noindent
Termin oddania w dziekanacie: ................\\[1cm]

\begin{center}
\begin{tabular}{lcr}
.............................................................. & ~~~ &
.............................................................. \\
(podpis kierownika katedry) & & (podpis opiekuna) \\
\end{tabular}
\end{center}

\newpage
\linespread{1.3}
\selectfont

\noindent
Na kolejnych dwóch stronach proszę dołączyć kolejno recenzje pracy popełnione przez Opiekuna oraz Recenzenta (wydrukowane z systemu MISIO i podpisane przez odpowiednio Opiekuna i Recenzenta pracy). Papierową wersję pracy (zawierającą podpisane recenzje) proszę złożyć w dziekanacie celem rejestracji.
\newpage
druga recenzja


\vspace{85mm}
\tableofcontents


\addcontentsline{toc}{chapter}{Wstep}
\chapter*{Wstęp}
Niniejsza praca opisuje aplikację ułatwiającą administratorowi tworzenie oraz administrację klasterm dla aplikacji WWW.\\
Została ona podzielona na pięć rozdziałów tworzących trzy grupy.\\
Pierwsza grupa skupia się na rozważaniach teoretycznych.
W jej skład wchodzą rozdziały:
\begin{description}
\item{"Wstęp do klastrowania"}\\
Opisuje on powody z których zaczęto stosować klastry WWW.
Przybliża podział klastrów oraz cechy każdego rodzaju.
Dodatkowo zarysowywuje problem wynikający z trudności zarządzania konfiguracją na klastrach.
\item{"Metody klastrowania"}
Rozdział ten opisuje najbardziej znane metody służące klastrowaniu aplikacji.
Opisuje on metody takie jak \textit{DNS round robin}, \textit{Nginx upstream}, \textit{Haproxy}, \textit{LSV}.
Każda z powyższych metod zostałą krótko opisana, sposób jej działania oraz sposób konfiguracji.
\item{"Zarządzanie konfiguracja"}\\
Ten rozdział przedstawia problem konsystentnej konfiguracji dużej ilości serwerów.
Opisuje on podstawowy, ręczny sposób konfiguracji maszyn.
Jego zalety i wady, a następnie kolejne metody usprawniające i eliminujące wady ręcznej konfiguracji.
Zostały opisane aplikacje takie jak: \textit{Fabric}, \textit{Puppet}, \textit{CFEngine}, \textit{Ansible}, ze szczególnym naciskiem na ostatni.
\end{description}

W drugiej części pracy, znajduje się jeden rozdział, opisujący praktyczne testy wydajnościowe opisywanych wcześniej metod klastrowania.
Skupia się on zarówno na testowaniu możliwości zwiększania wydajności jak i zapewnianiu wysokiej dostępności.

Trzecia część opisuje tytułowy \textit{System zautomatyzowanego zarządzania konfiguracją farmy serwerów aplikacji WWW}.
Opisuje on powody wyboru konkretnych technologii, opis ich konfiguracji oraz sposobu współdziałania.

\chapter{Wstęp do klastrowania}
\section{Idea klastrowania}
W czasach kiedy powstał internet, narodziła się potrzeba udostępniania informacji. Początkowo był to zwykle czysty tekst, który umieszczano na wtedy standardowych komputerach. Liczba odbiorców również nie była duża, co wynikało z dopiero tworzącej się sieci internet.

W miarę upływu czasu, strony internetowe zaczęły się rozrastać. Dodawano obrazy a następnie animacje i inne elementy poprawiające możliwości stron oraz odczucia wizualne odbiorców. Wzrosła również, w związku z coraz łatwiejszym dostępem do internetu, liczba użytkowników chcących uzyskać dostęp do stron internetowych.

Wymusiło to potrzebę używania coraz mocniejszych maszyn do serwowania treści. Niestety liczba użytkowników oraz wymagania stawiane stronom internetowym rosły szybciej niż postępował rozwój mocy obliczeniowych komputerów. Zaczęto używać dedykowanych serwerów dla aplikacji WWW zamiast zwykłych komputerów domowych. Jednak i to okazało się niewystarczające w obliczu wymagań stawianych przez dzisiejszy świat.

Aby rozwiązać ten problem, narodziła się idea połączenia kilku maszyn w jeden twór, tak aby zwiększyć całkowitą moc przeznaczoną na serwowanie treści.
\section{Rodzaje klastrów}
Istnieją dwa główne rodzaje klastrów:
\begin{itemize}
\item klastry wysokiej dostępności,
\item klastry wysokiej wydajności.
\end{itemize} 
Podział ten nie jest jednak bardzo sztywny.
Niektóre rzeczywiste klastry należą tylko do jednej grupy, jednak najczęściej mają one cechy obu tych grup.
\subsection{Klastry wysokiej dostępności}
Klastry wysokiej dostępności (ang. \textit{high availability, HA}) tworzone są głównie po to, aby zapewnić jak najwyższy poziom dostępności danej usługi.

Konstrukcje te składają się zwykle z wielu maszyn, z których jakaś część nie uczestniczy w serwowaniu danych, a jedynie czeka w gotowości i w przypadku awarii maszyn aktywnych przejmuje ich zadanie. Efektem jest ciągła dostępność usługi dla klienta.

Brak dostępności portalu może wiązać się z kosztami bądź brakiem zysków dla właściciela, dlatego często stosuje się klastry wysokiej dostępności.
\subsection{Klastry wysokiej wydajności}
Klastry wysokiej wydajności (ang. \textit{high performance, HP}) tworzone są głównie po to, aby zapewnić jak najwyższy poziom wydajności danej usługi.

Celem tego typu klastrów jest zapewnienie klientowi komfortu podczas korzystania z serwisu. W tej architekturze, pracują wszystkie maszyny, oraz każda z nim wykonuje jakąś część zadania w celu jak najszybszego skonstruowania odpowiedzi dla klienta.\\

W wersji ideowej, awaria jakiejś maszyny, unieruchamia klaster, ponieważ nie jest możliwe uzyskanie części odpowiedzi za którą odpowiadała maszyna która uległa awarii. W praktyce rzadko spotyka się czyste klastry wysokiej wydajności.
\subsection{Klastry mieszane}
Klastry mieszane są najczęściej spotykanymi klastrami. Posiadają one cechy obu powyższych grup, czyli najczęściej pracują wszystkie maszyny w klastrze, jednak ich konfiguracja pozwala im na wykonywanie dowolnego zadania (z puli przewidzianych zadań), w taki sposób, że podczas pracy wszystkich maszyn w klastrze, następuje szybsza odpowiedź do klienta - cecha wysokiej wydajności - jednak po awarii którejś z maszyn, pozostałe są w stanie przejąć jej obowiązki pozwalając odpowiedzieć na zapytanie - wysoka dostępność.\\
Można zauważyć, że podczas awarii węzłów w takim rodzaju klastra, spada wydajność, lecz zachowana jest ciągłość dostępu usługi, bo zwykle daje czas administratorowi na usunięcie usterki, bądź - jak zostanie pokazane w dalszej części tej pracy - skonfigurowanie nowych węzłów w celu zastąpienia tych które uległy awarii.
\section{Zarządzanie konfiguracją}
W czasach gdy strony były serwowane przez ich twórców na ich własnych komputerach, oni sami dbali o konfigurację swojej maszyny aby spełniała swoje zadanie.\\
W miarę jak zaczęto potrzebować coraz to większych mocy obliczeniowych, zaczęto wynajmować mocne serwery z dobrymi łączami, aby to one serwowały dane dla klientów.
Konfiguracją takiego serwera zajmował się twórca strony, bądź zatrudniony administrator.

Jednak, gdy zaczęto używać klastrów pojawił się problem ich konfiguracji. Gdy klaster składał się z kilku węzłów, było bardzo mozolną pracą skonfigurowanie każdego węzła osobno oraz pilnowanie, aby konfiguracje na każdym węźle były odpowiednie.
Po przypadkowej zmianie parametrów konfiguracji na jednej maszynie, trudno jest później znaleźć błąd.

Problem pojawił się również, gdy klastry zaczęły mieć więcej niż kilka węzłów. Skonfigurowanie kilkuset maszyn nie było prostym zadaniem, jak również prowadziło do wielu pomyłek. Dlatego administratorzy klastrów starali się ułatwić sobie pracę a zarazem uniknąć przypadkowych błędów.\\
Tak też powstały narzędzia do kontrolowania konfiguracji na wielu maszynach.

\chapter{Metody klastrowania}
W rozdziale tym przedstawię podstawowe metody wykorzystywane przy tworzeniu klastrów pod aplikacje internetowe.
\section{DNS round robin}
\subsection{Czym jest DNS}
DNS (ang {\textit{Domain Name System}) jest to system nazw domenowych. Usługa której najważniejszą funkcją jest przyporządkowanie nazwom domenowym (czytelnym dla człowieka) adresów IP.
Oznacza to, że system taki, jest w stanie zamienić nazwę \texttt{www.ftj.agh.edu.pl} na adres \texttt{149.156.110.3}.
Funkcjonalność taka w znacznym stopniu ułatwia korzystanie z sieci Internet, ponieważ przeciętnemu człowiekowi jest prościej zapamiętać mnemonik \texttt{www.ftj.agh.edu.pl} bądź \texttt{www.duckduckgo.com} niż ciągi czterech liczb.
Do zamiany nazwy domenowej na adres IP służą rekordy \texttt{A} oraz \texttt{AAAA}, wykorzystywane odpowiednio do adresów IPv4 oraz IPv6.
Przykład zastosowania rekordu \texttt{A} przedstawiam na poniższym listingu:
\lstinputlisting{lst/rr_arecord.lst}
w powyżej konfiguracji widzimy, że odpytując server DNS o wartość \texttt{nazwa4.domenowa.pl} otrzymamy informację, że dana nazwa wskazuje na adres \texttt{1.2.3.4}.\\
Innymi, często spotykanymi rekordami są rekordy \texttt{CNAME}, \texttt{MX} oraz \texttt{TXT}.
\begin{description}
\item[CNAME] jest to rekord będący wskaźnikiem. Dla przykładu:
\lstinputlisting{lst/rr_cname.lst}
widzimy, że w powyższym przykładzie \texttt{nazwa.domenowa.pl} wskazuje na 1.2.3.4.
Chcąc aby, \texttt{www.nazwa.domenowa.pl} również wskazywała w to samo miejsce, moglibyśmy również zdefiniować ją jako rekord \texttt{A} z takim samym adresem.
Jednak, w przypadku migracji serwera z adresu \texttt{1.2.3.4} na \texttt{1.2.3.5}, należałoby zmieniać ten adres w obu rekordach.
Zastosowanie rekordu \texttt{CNAME}, pozwala powiedzieć "nazwa \texttt{www.nazwa.domenowa.pl} wskazuje w to samo miejsce, w które wskazuje \texttt{nazwa.domenowa.pl}".
Prowadzi to do zmniejszenia ryzyka pomyłki przy wpisywaniu adresów IP, jak również zmniejsza liczbę miejsc w których należy zmienić adresację w przypadku migracji serwera.
\item[MX] rekord wskazujący na adres serwera pocztowego obsługującego daną domenę.
Strefa może zawierać kilka rekordów \texttt{MX} w celu dystrybucji ruchu na kilka serwerów pocztowych.
\lstinputlisting{lst/rr_mxrecord.lst}
Widzimy, że wpis definiujący rekord \texttt{MX} nie posiada nazwy. Zwykle rekord ten definiowany jest na początku definicji strefy, dlatego pominięcie nazwy powoduje, ze rekord ten odnosi się do nazwy tej strefy.
Jest to kolejna rzecz które uogólnia konfigurację i ułatwia migrowanie.
Wartość \texttt{5} oznacza poziom preferencji danego serwera. Mając zdefiniowanych kilka rekordów \texttt{MX}, poczta jest dystrybuowana przy pomocy algorytmu ważonego round robin.
\item[TXT] rekord który zgodnie z założeniami DNS miał zawierać dane tekstowe czytelne dla człowieka.
Obecnie rzadko zawiera dane dla użytkowników. Wykorzystywany jest głównie do konfiguracji SPF, co wykracza poza tematykę tej pracy.
\end{description}
\subsection{Opis metody}
Metoda ta polega na odpowiednim skonfigurowaniu strefy na serwerze DNS, w taki sposób, aby pod jedna nazwa rozwiązywała się na kilka adresów IP.
W efekcie, gdy serwer otrzyma zapytanie o daną nazwę domenową, zostanie mu zwrócona pula adresów zamiast jednego.
Aplikacja która otrzyma listę adresów IP, powinna połączyć się na losowy z nich.
Niestety nigdy nie ma pewności, że aplikacja posiada zaimplementowaną obsługę wielu adresów zwracanych przez serwer DNS, dlatego serwer wprowadza zabezpieczenie przed takim zachowaniem, a mianowicie tytułowy algorytm \textit{round robin}, który zwraca adresy IP, jednak za każdym razem ich permutację.\\
Dla przykładu, poniżej zamieszczone trzy zapytania wykonane po sobie.
\lstinputlisting{lst/rr_dig.shell}
Widzimy, że przy każdym zapytaniu, jako pierwszy adres zwracany jest kolejny adres z puli. Zapewnia to prawidłowe balansowanie ruchu, nawet przy aplikacjach nie potrafiących obsłużyć wielu adresów i łączących się na pierwszy otrzymany.

Metoda ta jest metodą wysokiej wydajności, ponieważ pozwala w sposób niewidoczny dla użytkownika rozdzielić ruch na kilka serwerów, a tym samym rozłożyć obciążenie, to skutkować będzie szybszą odpowiedzią klientowi na zapytanie.
Metoda ta nie zapewnia natywnie wykrywania niedostępności któregoś z serwerów, dlatego nie może służyć bezpośrednio jako metoda wysokiej dostępności.
Pośrednio występuje tutaj jednak mechanizm broniący przed niedostępnością któregoś z serwerów. W przypadku gdy aplikacja próbować się będzie połączyć z losowym adresem z puli, a połączenie nie będzie mogło być nawiązane, osiągnięty zostanie limit czasu połączenia (tzw. \textit{timeout}. W takiej sytuacji, dobrze napisana aplikacja, będzie próbować połączyć się na kolejny adres z puli, w nadziei, że będzie on dostępny.
W takiej sytuacji, połączenie zostanie nawiązane i klient otrzyma odpowiedź, jednak do czasu generowania odpowiedzi, trzeba doliczyć czas potrzebny na osiągnięcie \textit{timeout-u}. Może on wynieść od kilku, do kilkudziesięciu sekund.

Metoda ta jest również zależna od działania serwera DNS.
Najprostszym sposobem ochrony przed awarią tego systemu dystrybucji ruchu jest skonfigurowanie \textit{Secondary DNS}. To jednak wykracza poza tematykę tej pracy.
\subsection{Konfiguracja}
Konfiguracja DNS round robin jest stosunkowo prosta.
W konfiguracji strefy, należy umieścić wpis z wieloma rekordami \textit{A} dla jednej nazwy.
Przykład takiej strefy zamieszczony został poniżej
\lstinputlisting[caption=mgr.fabrykowski.pl.zone]{lst/rr_zone.zone}
W powyższym przykładzie, dla nazwy \texttt{rr.mgr.fabrykowski.pl} zostały zdefiniowane trzy adresy IP.
\section{Nginx}
\subsection{Czym jest nginx}
Nginx jest serwerem proxy oraz serwerem treści statycznych.
Wykorzystywany jest zwykle w połączeniu z serwerem Apache który serwuje treści PHP, podczas gdy sam dostarcza pliki statyczne (JavaScript, CSS, JPEG itp).
Drugim często wykorzystywanym modelem wykorzystania Nginx-a jest serwowanie treści statycznych oraz wykonywanie \textit{fastcgi pass}
\paragraph{Fastcgi pass} \hspace{0pt} \\
Moduł ten pozwala na komunikacje z procesami FastCGI.
Wykorzystanie FastCGI daje dużą niezależność w technologi opracowania aplikacji, która może zostać wykonana w PHP, Pythonie bądź Rubym.
Istnieje również możliwość zmiany wersji aplikacji, bądź technologii jej wykonania bez zmian w konfiguracji serwera, jeżeli aplikacja udostępnia to samo api FastCGI.

Do obsługi języka PHP zostanie wykorzystany php-fpm (\textit{PHP FastCGI Process Manager}.
Jest to alternatywna implementacja PHP FastCGI.
Pozwala ona na większą kontrolę w zakresie pul procesów - ich liczby oraz sposobu uruchamiania, jak również dowolność w kwestiach sieciowych - adres oraz port do nasłuchiwania.

Porównanie testów wydajności PHP-fpm oraz mod\_php do Apache jak również szybkość serwowania treści statycznych zostanie przedstawione w późniejszych rozdziałach.
\subsection{Opis metody}
Metoda klastrowania przy pomocy Nginx-a polega na zdefiniowaniu sekcji \texttt{upstream}.
Pozwala to na skonfigurowanie puli adresów do których będą przekazywane zapytania.
Aby dodać serwer do puli, należy podać jego adres IP bądź nazwę domenową oraz port.

Zapytania do serwerów wykonujących (\textit{workerów}) rozdzielane są równomiernie pomiędzy wszystkie serwery w puli.\\
Pozwala to na obsługiwanie zapytań na wielu maszynach, dlatego metoda ta pozwala na tworzenie klastrów \textbf{wysokiej wydajności}.

Ponadto, zaimplementowany jest również mechanizm sprawdzający stan poszczególnych serwerów w puli i w przypadku wykrycia awarii, oznaczany jest on jaki \textit{failure} i zapytanie nie są do niego kierowane.\\
Jest to zachowanie typowe dla klastrów \textbf{wysokiej dostępności}

Istnieje możliwość modyfikacji domyślnego algorytmu używanego przez Nginx-a.
\begin{itemize}
	\item zmiana sposobu dystrybucji zapytań dodając opcjonalny parametr \texttt{weight} mówiący o wadze danego węzła. Dla przykładu, w poniższej konfiguracji:
	\lstinputlisting{lst/nx_weight.conf}
	na każde 6 zapytań do \texttt{pula1}, 5 zostanie przekazanych do \texttt{server1} a jedno do \texttt{server2}.
	Opcja ta wykorzystywana jest głównie tam, gdzie poszczególne serwery różnią się parametrami bądź obciążeniem nie wynikającym z obsługiwania tej puli.
	\item zmiana sposobu określania serwera jako niedostępnego.
	Służą do tego parametry \texttt{max\_fails}, \texttt{fail\_timeout} oraz \texttt{slow\_start}.\\
	\begin{description}
	\item[\texttt{max\_fails}] określa liczbę nieudanych prób komunikacji z serwerem w czasie \texttt{fail\_timeout} nim serwer zostanie oznaczony jako niedostępny.
	Domyślna wartość tego parametru wynosi 1, natomiast wartość 0 wyłącza oznaczanie serwerów jako niedostępne.
	\item[\texttt{fail\_timeout}] określa czas w jakim musi nastąpić \texttt{max\_fails} nim serwer zostanie uznany za niedostępny.
	Określa również interwał czasowy co który będzie sprawdzana dostępność serwera.
	Wartość domyślna dla tego parametru wynosi 10 sekund.
	\item[\texttt{slow\_start}] określa czas w jakim będzie zwiększana wartość \texttt{weight} od zera do docelowej po przejściu serwera ze stanu niedostępnego do stanu dostępnego.
	Wartość domyślna wynosi 0, co oznacza wyłączone płynne włączanie serwera do puli.
	\end{description}
	\item oznaczenie konkretnych serwerów, jako serwery zapasowe.  
	Powoduje to nieprzekazywanie zapytań do tych serwerów jeżeli wszystkie serwery podstawowe odpowiadają.
	W przypadku, gdy któryś z podstawowych serwerów zostanie oznaczony jako niedostępny, zapytania zostają przekazywane do któregoś z serwerów zapasowych.
	Powoduje to zachowanie \textbf{wysokiej wydajności} oraz \textbf{wysokiej dostępności}.	
\end{itemize}
\subsection{Konfiguracja}
TODO
\section{Haproxy}
\subsection{Czym jest haproxy}
Haproxy jest serwerem proxy wysokiej dostępności (ang. High Availability Proxy).\\
Posiada on dwie główne funkcjonalności, które czynią go powszechnie używanym narzędziem.
Są nimi:\\
\begin{itemize}
	\item możliwość dystrybuowania ruchu na kilka maszyn, dając tym samym zwiększone możliwości obliczeniowe
	\item wykrywanie awarii serwerów \textit{backendowych} i nieprzekazywaniem do nich zapytań aż do czasu naprawy
\end{itemize}
\subsubsection{Funkcjonalność wysokiej wydajności}
Haproxy pozwala na zdefiniowanie tzw. \textit{backendu}, czyli grupy serwerów pełniących tą samą funkcje.
Decyzja o wyborze serwera dla danego zapytania może być podjęta na podstawie jednego z kilku algorytmów.
Poniżej znajduje się lista kilku najpopularniejszych. Pełną listę można znaleźć w dokumentacji\\
\LARGE TODO - bibliografia\normalsize
\begin{description}
	\item{Round robin}\\
		najpopularniejszy algorytm. Polega na rozdzielaniu zapytam do poszczególnych serwerów "po kolei".
		Kryterium modulującym działanie tego algorytmu jest parametr \texttt{weight}, który jak bardzo dany serwer ma być preferowany.
		Domyślna wartość \texttt{weight} wynosi 1. W przypadku, gdy wszystkie serwery mają takie same wartości, połączenia przekazywane są równo do każdego z nich.
	\item{Leastconn}\\
		wybór serwera podejmowany jest na podstawie ilości aktywnych połączeń do każdej maszyny.
		Wybierany jest serwer z najmniejszą ilością połączeń
	\item{Source}\\
		serwer docelowy wybierany jest na podstawie adresu nadawcy.
		Powoduje to, że jeden klient będzie zawsze obsługiwany przez tą samą maszynę. Pozwala to na uproszczenie obsługi sesji pomiędzy maszynami.
\end{description}
Ponieważ Haproxy działa w warstwie siódmej modelu OSI - czyli w warstwie aplikacji, możliwe jest również decydowanie o wyborze serwera na podstawie nagłówków zapytać HTTP.
Na podstawie np: wartości \texttt{host} w nagłówku HTTP, haproxy jest w stanie nasłuchując na jednym porcie dystrybuować ruch na odpowiednie \textit{backendy} odpowiedzialne za różne strony.
Przykłady takiego zastosowania zostaną przedstawione w podrozdziale "\ref{sec:haproxy_config} Konfiguracja"
\subsubsection{Funkcjonalność wysokiej dostępności}
Możliwości wykrywania niedostępności usługi oraz zapewnienia wysokiej dostępności były głównym celem twórców.
Świadczyć może o tym nazwa - \textit{HAProxy}, pochodzącą od \textit{High Availability} czyli wysoka dostępność.\\
Domyślnie haproxy nie sprawdza dostępności serwerów. Aby włączać tą funkcjonalność należy użyć parametru \texttt{check}.\\
\texttt{Check} sprawdza pod adresem i portem zdefiniowanymi dla danego serwera udaje się ustanowić połączenie TCP.
Jeśli tak jest, usługa jest uznawana za działającą i połączenia są kierowane na daną maszynę.\\
Istnieją również inne predefiniowane funkcje sprawdzające, np: \texttt{httpcheck} służący do sprawdzania odpowiedzi serwera WWW pod zadanym \texttt{uri}, \texttt{smtpcheck} - sprawdza usługę \texttt{smtp}, \texttt{mysql-check} oraz \texttt{pgsql-check} do baz danych.
Istnieje również możliwość stworzenia własnych mechanizmów sprawdzających działanie usługi, bazujące na technologii \texttt{expect}.
\subsection{Konfiguracja}
\label{sec:haproxy_config}
Na listingu  \ref{lst:haproxy_config} przedstawiono przykładową konfigurację HAProxy obsługującą zarówno wiele adresów url jak również stanowi \textit{frontend} dla \texttt{php-fpm}.
\lstinputlisting[caption=haproxy.cfg,label=lst:haproxy_config]{lst/haproxy_config.cfg}
W konfiguracji \textit{haproxy} wyróżniamy następujące ważne sekcje:
\begin{description}
	\item{\texttt{global}}\\
		zawiera konfigurację ustawień dla procesu haproxy.
		Umieszczane są tutaj informacje o ilości maksymalnych połączeń do procesu, \texttt{uid}-dzie bądź nazwie użytkownika i grupy z jakim należy uruchomić aplikacje, ścieżka do pliku z numerami \texttt{PID} procesów haproxy.
	\item{\texttt{defaults}}\\
		sekcja ta zawiera wartości domyślne dla innych sekcji.
		Pozwala to na umieszczenie dużej części konfiguracji w jednym miejscu, co ułatwia zarządzanie nią.
		Wartości zdefiniowane w sekcji \texttt{defaults} mogą być nadpisane w konkretnej sekcji wartością specyficzną dla danej sekcji.\\
		Dla przykładu, można zdefiniować domyślną wartość \texttt{timeout} na 10 sekund, natomiast dla pewnego serwisu, który wykonuje długotrwałe obliczenia, można tą wartość nadpisać wartości większa.
		Podejście takie pozwala na zachowanie zabezpieczenia przed nieodpowiadającymi procesami dla wszystkich serwisów, jednocześnie zezwalając aby serwis wykonujący długotrwałe obliczenia nie był anulowany przed uzyskaniem wyniku.
	\item{\texttt{listen}}\\
		definiuje usługę wysokiej dostępności. Po słowie kluczowym \texttt{listen} następuje nazwa danej usługi a następnie adres IP oraz port na którym będzie nasłuchiwać dana usługa.\\
		Słowo kluczowe \texttt{mode} definiuje w jakim trybie ma działać dana usługa.
		Wyróżniamy dwa tryby:
		\begin{itemize}
			\item \texttt{http}\\
				tryb ten działa w warstwie siódmej i pozwala na operowanie na zmiennych zawartych w nagłówkach HTTP
			\item \texttt{tcp}\\
				tryb ten działa w warstwie czwartej i powinien być stosowany do wszystkich połączeń nie będących połączeniami HTTP, tj. SSH, SSL i inne.
		\end{itemize}
		Historycznie istniał jeszcze tryb \texttt{health}, jednak jest już przestarzały i nie zalecane jest jego używanie.\\
		W przykładzie na listingu \ref{lst:haproxy_config} w usłudze \texttt{stats} znajdują się polecenia dające dostęp administratorowi do statystyk haproxy.
		W ich skład wchodzi m.in:
		\begin{itemize}
			\item ilość wszystkich połączeń przyjętych na dany \textit{frontend}
			\item ilość aktywnych połączeń na \textit{frontendach}
			\item stan serwerów obsługujących \textit{backendy}
			\item ilości połączeń na każdy serwer w \textit{backendach}
		\end{itemize}
		W przypadku większych instalacji, bądź potrzeby posiadania większej kontroli nad sposobem \textit{load balncingu} stosuje się strukturę rozbijającą prosty \texttt{listen} na dwie sekcje: \texttt{frontend} oraz \texttt{backend}.
	\item{\texttt{frontend}}\\
		\texttt{Frontend} odpowiedzialny jest za przyjmowanie i analizę połączeń od użytkownika.
		W sekcji tej znajduje znajduje się podzbiór poleceń z sekcji \texttt{listen} dotyczących opcji nasłuchiwania, takich jak adres oraz port, sposobu traktowania ruchu (http, tcp) jak również polecenia mogące analizować ruch w celu odpowiedniego jego obsłużenia.
		Noszą one nazwę \textit{ACL (and. Access Control List)}.
		Reguły ACL potrafią analizować ruch począwszy od warstwy czwartej do warstwy siódmej.\\
		Dla ruchu HTTP istnieje możliwość analizy nagłówków oraz rozdział połączań na różne serwery w zależności od pola \texttt{Host}, adresu \texttt{url} bądź metody żądania.
		Dla ruchu HTTPS istnieje możliwość konfiguracji ACL dla żądanego hosta dzięki technologi SNI \textit{ang. Server Name Indication} - technologia pozwalająca na odczytywanie wartości żądanego hosta w połączeniach HTTPS.
		Wykorzystywana w celu uruchamiania wielu adresów WWW na jednym porcie przy szyfrowaniu SSL.\\
		Po zdefiniowaniu ACL, istnieje możliwość zdefiniowania użycia konkretnego \texttt{backend}-u w zależności od przypisania do ACL.\\
		Dodatkowo, oprócz wyboru odpowiedniego \texttt{backend}-u istnieje możliwość odrzucania połączeń spełniających warunki ACL - np: zbyt duża ilość połączeń.
	\item{\texttt{backend}}\\
		Sekcja ta definiuje zaplecze serwerowe, tj. listę serwerów do których ma być kierowany ruch.
		Są tutaj w mocy wszystkie polecenia które mogą znaleźć się w sekcji \texttt{listen} które tyczą się serwerów obsługujących, czyli m.in. algorytm rozdziału połączeń czy specyficzne metody sprawdzania dostępności serwera.
\end{description}
\section{LVS}
\subsection{Czym jest LVS}
LVS (\textit{ang. Linux Virtual Server}) jest technologią pozwalającą na tworzenie klastrów bazujących na systemach GNU/Linux.
Metoda ta jest bardzo wysoko skalowalna przy małym obciążeniu procesora.\\
Trzeba mieć na uwadze, że LVS nie możliwości współbieżnego przetwarzania operacji, a jedynie dystrybucję połączeń pomiędzy wiele serwerów.
\subsection{Opis metody}
LVS jest technologią działającą w czwartej warstwie modelu OSI, tj. w warstwie transportowej.
Zakłada ona istnienie dwóch typów serwerów w klastrze:
\begin{description}
	\item{\textit{Director}}\\
		Jest to serwer zarządzający. Występuje jeden w klastrze.
		To do niego kierowane są połączenia klienckie.
	\item{\textit{Real Server}}\\
		Jest to serwer z właściwą usługą. W klastrze może występować ich wiele.
		Odpowiedzialne są za przetwarzanie zapytań od klienta.
\end{description}
LVS może działać w jednym z trzech trybów:
\begin{description}
	\item{\textit{NAT}}\\
		W tym trybie zapytania przychodzące od klienta do \textit{Directora} zostają znatowane na adres jednego z \textit{Real server}-ów.
		Po obsłużeniu zapytania, \textit{real server} przekazuje odpowiedź do \textit{director}-a który następnie przekazuje odpowiedź do klienta.\\
		Tłumaczenie pakietów wymaga pewnej mocy obliczeniowej, ponadto \textit{director} uczestniczy w przesyłaniu zapytania oraz odpowiedzi, co sprawia, że tryb NAT ma ograniczoną skalowalność ograniczoną mocą procesora oraz łącza internetowego.
	\item{\textit{Direct Routing}}\\
		Tryb ten jest wolny od problemów z mocą obliczeniową oraz utylizacją łącza występujących w przypadku trybu NAT.\\
		W przypadku \textit{direct routing}-u \textit{real server}-y posiadają dodatkowe adresy IP, takie same jak adresy IP używane do \textit{load balancingu} na \textit{directorze} jednak \textit{real server} musi być tak skonfigurowany aby nie odpowiadał na zapytania ARP o te adresy.\\
		Gdy pakiet dochodzi do \textit{directora}, zostaje on przekazany w niezmienionej formie (od trzeciej warstwy wzwyż) do jednego z \textit{real server}-ów. \textit{Director} musi mieć możliwość bezpośredniego połączenia z \textit{real server}-em aby móc opakować pakiet w odnowienie nagłówki warstwy drugiej.
		Ponieważ \textit{real server} posiada dodatkowy adres IP, pakiet przekazany przez \textit{director}-a jest akceptowany ponieważ adres docelowy w pakiecie zgadza się z adresem posiadanym przez \textit{real server}.
		Po przyjęciu pakiety, \textit{real server} odpowiada na niego na adres źródłowy zawarty w pakiecie, czyli bezpośrednio do klienta - omijając \textit{director}-a.
		Następnie, klient wysyłając kolejne kieruje je do \textit{director}-a, który śledząc połączenia, przekazuje je zawsze do odpowiedniego \textit{real server}-a.

		Powyższa procedura powoduje, iż \textit{director} nie jest obciążany obliczaniem adresacji dla NAT, gdyż przekazuje pakiety w niezmienionej formie, oraz zmniejsza utylizacje łącza, ponieważ na łączy \textit{director}-a przesyłane są jedynie pakiety z żądaniami (na wejściu - od klienta, i na wyjściu - do \textit{real server}-a), natomiast pakiety z odpowiedzią, np: HTTP, które są zwykle znacznie większe niż zapytania, są przesyłane łączami używanymi przez \textit{real server}-y.
		W efekcie, metoda ta jest wysoko skalowalna.
	\item{\textit{IP tunneling}}\\
		Tryb ten działa identycznie jak \textit{direct routing}, z tą różnicą, ze \textit{director} oraz \textit{real server}-y nie znajdują się w jednej fizycznej sieci, lecz są spięte jakimś tunelem.
\end{description}

Należy zwrócić uwagę, że LVS nie posiada żadnego wbudowanego systemu zapewniającego wysoką dostępność.
Jest to metoda zapewniająca wyłącznie wysoką wydajność.
Istnieją rozwiązania współpracujące z LVS dodające funkcjonalność wykrywania niedostępności usługi na \textit{real server}-ach i wypinające je z konfiguracji LVS.\\
Te rozwiązania są jednak poza zakresem niniejszej pracy.
\subsection{Konfiguracja}

\chapter{Zarządzanie konfiguracją}
W rozdziale tym przedstawię różne metody zarządzania konfiguracją serwerów. Postaram się opisać poglądowo różne metody, jak również przedstawić zalety i wady poszczególnych z nim.
\section{Ręczna konfiguracja każdego serwera za pomocą SSH}
\subsection{Opis}
Ręczna konfiguracja serwerów stos osuwana jest głównie tam, gdzie administrator ma pod swoją opieką jeden bądź kilka serwerów. W takim przypadku zmiana konfiguracji na serwerze jest prosta i nie zajmuje dużej ilości czasu.\\
Konfiguracja taka nie wymaga od administratora żadnej wiedzy wykraczającej poza obszar konfigurowanego systemu oraz usług, a wprowadzane zmiany widoczne są od razu po wprowadzeniu.
Ten sposób konfiguracji spotykany jest czasem w większych systemach informatycznych.
Dzieje się tak zwykle w jednostkach szybko rozwijających się, gdzie nastąpił szybki wzrost liczby serwerów i nie opracowano jeszcze metoda automatyzacji konfiguracji.

Do konfiguracji ręcznej nie potrzeba żadnego dodatkowego oprogramowania ani po stronie maszyn konfigurowanych, ani maszyny z której następuje konfiguracja.
Na maszynie z której następuje konfiguracja musi być dostępny klient SSH, który jest instalowany domyślnie we wszystkich dystrybucjach systemów GNU/Linux, a na maszynach konfigurowanych musi być zainstalowany i uruchomiony serwer SSH - jest on domyślnie zainstalowany w większości dystrybucji serwerowych GNU/Linux i w części dystrybucji przeznaczonych na komputery domowe.

Wadą takiej metody jest również sytuacja, w której tylko jedna osoba, bądź mała grupa osób, zna konfigurację poszczególnych serwerów oraz usług.
W przypadku opuszczenia przez daną osobę zespołu, pozostali członkowie muszą, analizując pliki konfiguracyjne, zrozumieć zamysł osoby to tworzącej.\\
Kolejną wadą, jest brak możliwości powielenia konfiguracji.
W przypadku gdy zaistnieje potrzeba skonfigurowania bliźniaczego serwera, jako serwera zapasowego, należy każdą usługę skonfigurować od nowa na wzór serwera pierwotnego. Również wprowadzane zmiany należy uwzględniać na wszystkich serwerach.
Może to w prosty sposób prowadzić do błędów i rozbieżności konfiguracji.
\subsection{Zalety i wady}
Zalety:
\begin{itemize}
\item prostota
\item używanie tylko domyślnych komponentów systemu
\item szybkość wprowadzanych zmian
\item informacja zwrotna czy usługa została uruchomiona poprawnie
\end{itemize}
Wady:
\begin{itemize}
\item \textbf{brak skalowalności}
\item różnice między poszczególnymi serwerami
\item trudność powielania
\item wiedza o konfiguracji zależna od jednego pracownika
\end{itemize}
\subsection{Przykład}
\lstinputlisting{lst/conf_ssh.sh}
\subsection{CSSH}
Istnieje narzędzie CSSH (\textit{Cluster SSH}) które wychodzi na przeciw osobą chcącym konfigurować kilka serwerów jednocześnie poprzez SSH.
Narzędzie to potrafi otworzyć wiele sesji SSH równolegle - każda sesja w osobnym terminalu.
Głównym interface-em programu, jest małe okno wejścia, które przechwytując wpisywany do niego tekst, przesyła go do wszystkich otwartych sesji.\\
Zmniejsza to prawdopodobieństwo rozbieżności w konfiguracji, jak również przyśpiesza proces, ponieważ tekst jest wpisywany do wszystkich sesji jednocześnie i nie ma potrzeby wielokrotnego wpisywania tej samej konfiguracji na wielu maszynach.\\
Aplikacja umożliwia również przełączenie się w dowolnej chwili na konkretny terminal i interakcję tylko z jednym serwerem, np: w celu zdiagnozowania problemu występującego tylko na tej jednej maszynie.
\section{Fabric}
\subsection{Opis}
Jest aplikacją napisaną w języku Python, służącą głównie do wykonywania poleceń powłoki na zdalnym serwerze. Aplikacja pozwala na zdefiniowanie kolejności w jakiej mają zostać poszczególne polecenia, jak również udostępnia kilka funkcji sprawdzających, np: czy plik istnieje, bądź kopiowanie plików na lub z serwera.\\
Sprawdza się wszędzie tam, gdzie chcemy wykonać konkretne operacje na zdalnym systemie niezależnie od aktualnego stanu tego systemu, bądź z niewielkim wpływem obecnych czynników.
Zastosowanie fabrica można porównać do CSSH, z tą różnicą, że operacje nie są wpisywane przez administratora podczas sesji, a zdefiniowane wcześniej w pliku, co w znacznym stopniu ułatwia powtarzalność wykonywania zdefiniowanych operacji.
Pozwala również w prosty sposób rozdzielić zdefiniowane zadania na poszczególne grupy serwerów na których należy je wykonać.\\
Typowe zastosowania:
\begin{itemize}
\item restart nietypowych usług nie posiadających jeszcze odpowiednich skryptów sysvinit
\item rekonfiguracja projektów na zdalnych serwerach po wysłaniu zmian przez system kontroli wersji
\item przeszukiwanie logów poszczególnych serwerów
\end{itemize}
\subsection{Zalety i wady}
Zalety:
\begin{itemize}
\item łatwość instalacji - repozytoria dystrybucji oraz pythonowe
\item równoległe wykonywanie operacji
\item łatwość konfiguracji
\item powtarzalność wykonywania
\item skalowalność
\end{itemize}
Wady:
\begin{itemize}
\item ograniczone możliwości decyzji na podstawie aktualnej konfiguracji
\item wykonywanie tylko poleceń powłoki
\end{itemize}
\subsection{Przykład}
\lstinputlisting[language=python,caption=fabfile.py]{lst/conf_fabfile.py}
przykład działania powyższego skryptu:
\lstinputlisting{lst/conf_fabfile_run.sh}
Aplikacja została uruchomiona z parametrami:
\begin{description}
\item[-P] równolegle wykonywanie zadań
\item[-z 5] uruchomienie pięciu równoległych połączeń
\item[-I] zapytanie o hasło do serwerów (używane gdy niedostępne logowanie po kluczach SSH)
\item[show\_problem] nazwa zadania zdefiniowana w pliku \texttt{fabfile.py}
\end{description}
Fabric wykonuje połączenia do hostów zdefiniowanych w zmiennej \texttt{env.hosts} w liczbie pięciu połączeń równoległych.
W przypadku nie podania parametru \texttt{-z}, aplikacja wykona liczbę równoległych połączeń równą liczbie zdefiniowanych hostów dla danego zadania.\\
Po połączeniu się do zdalnego hosta, następuje sprawdzenie czy istnieje plik \texttt{/var/problem}. W przypadku wykrycia istnienia takiego pliku, zostaje wywołane polecenie powłoki \texttt{cat}.
W wyniku wykonywania widzimy, ze plik \texttt{/var/problem} istniał tylko na serwerze o adresie IP \texttt{192.168.0.12} i zawierał tekst \textit{zasob byl podmontowany}.

\chapter{Testowanie rozwiązania}
\section{Środowisko testowe}
Wszystkie rozwiązania testowane będą przy użyciu następującego środowiska testowego:
\begin{description}
\item{Maszyna fizyczna:}
    \begin{itemize}
	\item CPU: \texttt{Intel(R) Core(TM)2 Quad CPU    Q9400  @ 2.66GHz} posiadający wsparcie dla wirtualizacji (\textit{VT-x})
	\item Pamięć RAM: \texttt{8G DDR2}
%%	\item OS: \texttt{Linux redraptor 3.18.1-gentoo \#7 SMP Wed Dec 31 02:04:37 CET 2014 x86\_64}
	\item OS: \texttt{Gentoo Linux 64bit, kernel 3.18.1}
	\item Platforma wirtualizacyjna: \texttt{KVM} (host)
    \end{itemize}
\item{Maszyna wirtualna na potrzeby kontenerów:}
    \begin{itemize}
	\item CPU: Mapowany z maszyny fizycznej. Przydział 3 rdzeni
	\item Pamięć RAM: \texttt{2G}
%%	\item OS: \texttt{Linux lxc 3.13.0-24-generic \#46-Ubuntu SMP Thu Apr 10 19:11:08 UTC 2014 x86\_64 x86\_64 x86\_64}
	\item OS: \texttt{Ubuntu Linux 64 bit, kernel 3.13.0-24-generic}
	\item Platforma wirtualizacyjna: \texttt{KVM} (guest), \texttt{LXC} (host)
    \end{itemize}
\item{Kontenery \texttt{LXC} do celów testowania aplikacji:}
    \begin{itemize}
	\item OS: Ubuntu linux. Jądra współdzielone z maszyną hostującą.
	\item Ustawienia \texttt{cgroups}: \texttt{lxc.cgroup.cpu.cfs\_quota\_us = 30000}
    \end{itemize}
\item{Maszyna wirtualna na potrzeby \texttt{LVS}:}
    \begin{itemize}
	\item CPU: Mapowany z maszyny fizycznej. Przydział 1 rdzeń
%%    	\item OS: \texttt{Linux mgr10 3.13.0-24-generic \#46-Ubuntu SMP Thu Apr 10 19:11:08 UTC 2014 x86\_64 x86\_64 x86\_64}
	\item OS: \texttt{Ubuntu Linux 64 bit, kernel 3.13.0-24-generic}
	\item Pamięć RAM: \texttt{192M}
    \end{itemize}
\end{description}
Aplikacje działające w \texttt{userspace}, tj. \texttt{apache, nginx, haproxy, php-fpm}, zostają uruchamiane w dedykowanych kontenerach \texttt{LXC}.
Usługi działające w warstwie jądra, tj. \texttt{LVS} zostają uruchomione na dedykowanej maszynie wirtualnej przy użyciu \texttt{KVM}.
\section{Wybór serwera WWW}
W rozdziale tym zostanie przedstawione zestawienie kilku testów wydajnościowych dwóch serwerów WWW.
\begin{itemize}
\item Apache2
\item Nginx
\end{itemize}
Przetestowane zostanie serwrowanie plików statycznych oraz treści dynamicznych PHP{.}\\
Wszystkie testy zostały przeprowadzone z wykorzystaniem 10 000 połączeń.\\
Wszystkie czasy zostały podane w milisekundach.
\subsection{Pliki statyczne}
Testy plików statycznych przeprowadzone zostaną przy użyciu dwóch plików HTML{.}
Jeden o rozmiarze 10 bajtów, drugi o rozmiarze 100 kilobajtów.

\begin{figure}
	\centering
	\begin{subfigure}[h]{0.3\textwidth}
		\includegraphics[width=\textwidth]{testy/wybor_index_maly_1.png}
		\caption{1 równoległe zapytanie}
	\end{subfigure}
	\begin{subfigure}[h]{0.3\textwidth}
		\includegraphics[width=\textwidth]{testy/wybor_index_maly_2.png}
		\caption{2 równoległe zapytania}
	\end{subfigure}
	\begin{subfigure}[h]{0.3\textwidth}
		\includegraphics[width=\textwidth]{testy/wybor_index_maly_4.png}
		\caption{4 równoległe zapytania}
	\end{subfigure}

	\begin{subfigure}[h]{0.3\textwidth}
		\includegraphics[width=\textwidth]{testy/wybor_index_maly_8.png}
		\caption{8 równoległych zapytań}
	\end{subfigure}
	\begin{subfigure}[h]{0.3\textwidth}
		\includegraphics[width=\textwidth]{testy/wybor_index_maly_16.png}
		\caption{16 równoległych zapytań}
	\end{subfigure}
	\begin{subfigure}[h]{0.3\textwidth}
		\includegraphics[width=\textwidth]{testy/wybor_index_maly_32.png}
		\caption{32 równoległe zapytania}
	\end{subfigure}

	\begin{subfigure}[h]{0.3\textwidth}
		\includegraphics[width=\textwidth]{testy/wybor_index_maly_64.png}
		\caption{64 równoległe zapytania}
	\end{subfigure}
	\begin{subfigure}[h]{0.3\textwidth}
		\includegraphics[width=\textwidth]{testy/wybor_index_maly_128.png}
		\caption{128 równoległe zapytania}
	\end{subfigure}
	\begin{subfigure}[h]{0.3\textwidth}
		\includegraphics[width=\textwidth]{testy/wybor_index_maly_256.png}
		\caption{256 równoległych zapytań}
	\end{subfigure}
	\caption{Zapytania o mały plik statyczny}\label{fig:wyb_index_maly}
\end{figure}
\begin{figure}
	\centering
	\begin{subfigure}[h]{0.3\textwidth}
		\includegraphics[width=\textwidth]{testy/wybor_index_duzy_1.png}
		\caption{1 równoległe zapytanie}
	\end{subfigure}
	\begin{subfigure}[h]{0.3\textwidth}
		\includegraphics[width=\textwidth]{testy/wybor_index_duzy_2.png}
		\caption{2 równoległe zapytania}
	\end{subfigure}
	\begin{subfigure}[h]{0.3\textwidth}
		\includegraphics[width=\textwidth]{testy/wybor_index_duzy_4.png}
		\caption{4 równoległe zapytania}
	\end{subfigure}

	\begin{subfigure}[h]{0.3\textwidth}
		\includegraphics[width=\textwidth]{testy/wybor_index_duzy_8.png}
		\caption{8 równoległych zapytań}
	\end{subfigure}
	\begin{subfigure}[h]{0.3\textwidth}
		\includegraphics[width=\textwidth]{testy/wybor_index_duzy_16.png}
		\caption{16 równoległych zapytań}
	\end{subfigure}
	\begin{subfigure}[h]{0.3\textwidth}
		\includegraphics[width=\textwidth]{testy/wybor_index_duzy_32.png}
		\caption{32 równoległe zapytania}
	\end{subfigure}

	\begin{subfigure}[h]{0.3\textwidth}
		\includegraphics[width=\textwidth]{testy/wybor_index_duzy_64.png}
		\caption{64 równoległe zapytania}
	\end{subfigure}
	\begin{subfigure}[h]{0.3\textwidth}
		\includegraphics[width=\textwidth]{testy/wybor_index_duzy_128.png}
		\caption{128 równoległe zapytania}
	\end{subfigure}
	\begin{subfigure}[h]{0.3\textwidth}
		\includegraphics[width=\textwidth]{testy/wybor_index_duzy_256.png}
		\caption{256 równoległych zapytań}
	\end{subfigure}
	\caption{Zapytania o duży plik statyczny}\label{fig:wyb_index_duzy}
\end{figure}
Wykresy na rys.~\ref{fig:wyb_index_maly} przedstawiają czasy obsłużenia zapytań o plik HTML o rozmiarze 10 bajtów, natomiast wykresy na rys.~\ref{fig:wyb_index_duzy} czasy zapytań o plik o rozmiarze 100 kilobajtów.\\
Można zauważyć, że wyniki dla małych plików statycznych są zbliżone zarówno dla Apache jak i Nginx z lekką przewagą dla Nginx.
Największą przewagę Nginx-a widać przy średnim i dużym obciążeniu.
\clearpage
\subsection{Treść dynamiczna}
Testy treści dynamicznej przeprowadzane są przy użyciu konfiguracji Nginx + php-fpm oraz Apache + php-fpm.
Konfiguracja Apache + mod\_php została odrzucona, ponieważ wymaga umieszczenia serwera WWW oraz serwera PHP na jednym serwerze, co uniemożliwia użycie wielu serwerów PHP dla jednego serwera WWW.\\
Do testów zostały wykorzystane dwa bliźniacze skrypty obliczające liczby ciągu Fibonacciego.
Jeden ze skryptów został przedstawiony na listingu~\ref{lst:fib}.
Obliczane są wyrazy: piąty --- dla skryptu wykonującego się szybko, oraz piętnasty --- dla skryptu wykonującego się dłużej.\\
Wykorzystany został model obliczanie wartości rekurencyjny, ponieważ w przeciwieństwie do iteracyjnego wymaga większej mocy obliczeniowej.
Jest to pożądane aby czas obsługi zapytania obejmował czas wykonywania skryptu, a nie jedynie obsługi sesji HTTP oraz transferu danych (zostało to przetestowane przy wykorzystaniu \textit{szybkiego skryptu}).
\lstinputlisting[caption=fib.php,label=lst:fib,language=php]{lst/fib.php}
\begin{figure}
	\centering
	\begin{subfigure}[h]{0.3\textwidth}
		\includegraphics[width=\textwidth]{testy/wybor_fib_5_1.png}
		\caption{1 równoległe zapytanie}
	\end{subfigure}
	\begin{subfigure}[h]{0.3\textwidth}
		\includegraphics[width=\textwidth]{testy/wybor_fib_5_2.png}
		\caption{2 równoległe zapytania}
	\end{subfigure}
	\begin{subfigure}[h]{0.3\textwidth}
		\includegraphics[width=\textwidth]{testy/wybor_fib_5_4.png}
		\caption{4 równoległe zapytania}
	\end{subfigure}

	\begin{subfigure}[h]{0.3\textwidth}
		\includegraphics[width=\textwidth]{testy/wybor_fib_5_8.png}
		\caption{8 równoległych zapytań}
	\end{subfigure}
	\begin{subfigure}[h]{0.3\textwidth}
		\includegraphics[width=\textwidth]{testy/wybor_fib_5_16.png}
		\caption{16 równoległych zapytań}
	\end{subfigure}
	\begin{subfigure}[h]{0.3\textwidth}
		\includegraphics[width=\textwidth]{testy/wybor_fib_5_32.png}
		\caption{32 równoległe zapytania}
	\end{subfigure}

	\begin{subfigure}[h]{0.3\textwidth}
		\includegraphics[width=\textwidth]{testy/wybor_fib_5_64.png}
		\caption{64 równoległe zapytania}
	\end{subfigure}
	\begin{subfigure}[h]{0.3\textwidth}
		\includegraphics[width=\textwidth]{testy/wybor_fib_5_128.png}
		\caption{128 równoległe zapytania}
	\end{subfigure}
	\begin{subfigure}[h]{0.3\textwidth}
		\includegraphics[width=\textwidth]{testy/wybor_fib_5_256.png}
		\caption{256 równoległych zapytań}
	\end{subfigure}
	\caption{Zapytania o szybki skrypt PHP}\label{fig:wyb_fib_5}
\end{figure}
\begin{figure}
	\centering
	\begin{subfigure}[h]{0.3\textwidth}
		\includegraphics[width=\textwidth]{testy/wybor_fib_15_1.png}
		\caption{1 równoległe zapytanie}
	\end{subfigure}
	\begin{subfigure}[h]{0.3\textwidth}
		\includegraphics[width=\textwidth]{testy/wybor_fib_15_2.png}
		\caption{2 równoległe zapytania}
	\end{subfigure}
	\begin{subfigure}[h]{0.3\textwidth}
		\includegraphics[width=\textwidth]{testy/wybor_fib_15_4.png}
		\caption{4 równoległe zapytania}
	\end{subfigure}

	\begin{subfigure}[h]{0.3\textwidth}
		\includegraphics[width=\textwidth]{testy/wybor_fib_15_8.png}
		\caption{8 równoległych zapytań}
	\end{subfigure}
	\begin{subfigure}[h]{0.3\textwidth}
		\includegraphics[width=\textwidth]{testy/wybor_fib_15_16.png}
		\caption{16 równoległych zapytań}
	\end{subfigure}
	\begin{subfigure}[h]{0.3\textwidth}
		\includegraphics[width=\textwidth]{testy/wybor_fib_15_32.png}
		\caption{32 równoległe zapytania}
	\end{subfigure}

	\begin{subfigure}[h]{0.3\textwidth}
		\includegraphics[width=\textwidth]{testy/wybor_fib_15_64.png}
		\caption{64 równoległe zapytania}
	\end{subfigure}
	\begin{subfigure}[h]{0.3\textwidth}
		\includegraphics[width=\textwidth]{testy/wybor_fib_15_128.png}
		\caption{128 równoległe zapytania}
	\end{subfigure}
	\begin{subfigure}[h]{0.3\textwidth}
		\includegraphics[width=\textwidth]{testy/wybor_fib_15_256.png}
		\caption{256 równoległych zapytań}
	\end{subfigure}
	\caption{Zapytania o wolny skrypt PHP}\label{fig:wyb_fib_15}
\end{figure}
Na wykresach~\ref{fig:wyb_fib_5} oraz~\ref{fig:wyb_fib_15} obrazujących czasy obsługi zapytań do skryptów PHP, można zauważyć że różnice pomiędzy Apache a Nginx są mniejsze niż dla plików statycznych.
Wynika to z faktu, ze obsługą zapytań w obu przypadkach zajmuje się PHP-fpm, natomiast serwer WWW odpowiedzialny jest jedynie za przekazywanie zapytań do \textit{backendu}.
\subsection{Podsumowanie}
Jak wykazały testy, Nginx daje krótsze czasy odpowiedzi we wszystkich testowanych sytuacjach, dlatego został wybrany jako podstawowy serwer wykorzystywany w przedstawionym projekcie.

\chapter{Opis projektu}
\section{Opis}
W poprzednich rozdziałach zostały przedstawione metody klastrowania oraz zarządzania konfiguracją, a następnie zostały one przetestowane.
Została również wykazana zasadność ich stosowania.\\
Jednak, aby wdrożyć takie rozwiązania, potrzebna jest wiedza oraz czas pracy administratora.

\textit{System zautomatyzowanego zarządzania konfiguracją farmy serwerów aplikacji WWW} (zwany dalej \textit{SZZ}) ma za zadanie uprościć konfigurację klastra WWW, pozwalając zaoszczędzić czas i pieniądze.
Opisywany system będzie mógł być obsługiwany przez osoby nie posiadające dogłębnej wiedzy z zakresu administracji systemami linux ani serwerami WWW.
Wymagana jest jedynie podstawowa wiedza techniczna, którą posiada przeciętny programista.

Konfiguracja odbywa się poprzez edycję plików, dlatego obsługująca system powinna być w stanie obsługiwać połączenia \texttt{ssh} oraz edytor tekstowy.
\section{Struktura}
\textit{SZZ} wykorzystuje różne metody klastrowania.
Struktura systemu została przedstawiona na rys~\ref{fig:struktura}
\begin{figure}
	\centering
	\includegraphics[width=\textwidth]{obrazy/struktura_szz.png}
	\caption{Struktura SZZ}
	\label{fig:struktura}
\end{figure}
Pierwszą warstwą klastrowania jest LVS (por.~\ref{sec:LVS}).
Zapytanie trafiające na serwer obsługiwane są przez \textit{direcotor}-a.
Następnie przekazywane są do serwerów WWW, które analizując zapytania serwują treści statyczne ze wspólnego zasobu NFS.
W przypadku zapytania o treści dynamiczne, zapytania przekazywane jest do warstwy trzeciej projektu, czyli usługi Haproxy, która przekazuje zadania do odpowiedniego serwera z usługą \textit{PHP-fpm}.
Po wygenerowaniu odpowiedzi, serwer roboczy zwraca odpowiedź do Haproxy, które przekazuje je do serwera WWW.
Ten natomiast, będąc \textit{real server}-em w klastrze LVS, odpowiada bezpośrednio klientowi.
\subsection{Warstwa zero - storage}
Projekt posiada wspólny zasób dyskowy wystawiany poprzez protokół NFS.
Metoda ta w znaczny sposób uprasza aktualizację aplikacji na wszystkich węzłach równocześnie - kosztem braku możliwości wykonywania tzw. \textit{rolling update}.
Ta technologi rozwiązuje również problem wgrywania plików na serwer oraz ich propagacji ponieważ każdy wgrywany plik trafia na wspólny zasób i jest od razu widziany przez pozostałe węzły.\\
Wydajność NFS jest zadowalająca przy wykorzystywaniu w obrębie jeden serwerowni i jednej sieci LAN.
W przypadku chęci użycia rozproszenia systemu między kilkoma \textit{datacenter} należy ze własnym zakresie obsłużyć synchronizację wgrywanych plików oraz aktualizacji.
\subsection{Warstwa pierwsza - LVS}
Jedynym wystawionym na świat serwerem jest \textit{director}. Do niego trafiają wszystkie zapytania od klientów.
Wykorzystana w projekcie konfiguracja używa \textit{scheduler}-a opartego o algorytm \textit{round robin}, czyli przekazuje zapytania na wszystkie serwery po kolei.
Technologia LVS pozwala na posiadanie tylko jednego serwera typu \textit{director}, ponieważ jego zadaniem jest jedynie przekazywanie zapytać do \textit{real server}-ów.
Ponadto, jak zostało omówione wcześniej, odpowiedzi do klienta wysyłane są bezpośrednio od \textit{real server}-ów, bez udziału \textit{director}-a co pozwala na obsługę nawet dużego ruchu.\\
Obecna konfiguracja nie posiada narzędzi do wykrywania niedostępności \textit{director}-a bądź \textit{real server}-ów, dlatego konfiguracja narzędzi typu \textit{heartbeat} oraz technologi \textit{floating IP} i/lub monitoringu stanu serwerów, leży po stronie użytkownika.
\subsection{Warstwa druga - Nginx}
Drugą warstwą jest warstwa serwerów WWW.
Do nich trafiają zapytania przekazywane z pierwszej warstwy.
Serwer WWW obsługujący wiele \textit{Virtual Host}-ów, analizuje zapytanie pod kontem, czy żądana ścieżka jest istniejącym plikiem na dysku.
Jeżeli plik istnieje, jest on serwowany klientowi.
W przeciwnym wypadku, zapytanie zostaje przekazywane do haproxy.
\subsection{Warstwa trzecia - Haproxy}
Haproxy jest trzecią warstwą systemu.
Przez tą warstwę przechodzą wszystkie zapytania o treści dynamiczne.
Usługa tworzy osobny \textit{frontend} oraz \textit{backend} dla każdego projektu.\\
Haproxy posiada wbudowaną obsługę wykrywania, dlatego warstwa trzecia zapewnia pełna \textit{HA}.\\
Wysycenie łącza dla warstwy trzeciej nie powinna być problemem, ponieważ zapytania odbywają się jedynie po dane dynamiczne - zwykle tekstowe.
Wszystkie zapytania o obrazy i inne treści statyczne zostają obsłużone warstwę wcześniej.
System nie zapewnia wysokiej dostępności dla usługi haproxy.
Administrator powinien skonfigurować monitoring aby móc taką awarię wykryć maksymalnie szybko i usunąć usterkę.
W przypadku niemożliwości naprawy maszyny, system pozwala na skonfigurowanie nowej maszyny dla warstwy trzeciej oraz przekonfigurowanie w stosunkowo krótkim czasie.
\subsection{Warstwa czwarta - PHP-fpm}
Najniższą warstwą systemu jest warstwa robocza.
PHP-fpm odpowiedzialny jest za generowanie treści dynamicznych.
Podobnie jak serwer WWW, korzysta on ze współdzielonego zasobu dyskowego udostępnianego po NFS.
Na jednej maszynie może być uruchomionych kilka aplikacji PHP-fpm.
\section{Nazwa robocza: backend}
Sekcja ta opisuje kroki jakie podejmuje system, aby skonfigurować klaster zgodnie z założeniami.
\subsection{NFS}
Aby skonfigurować serwer NFS, system instaluje potrzebne pakiety a następnie kopiuje plik konfiguracyjny na serwer.
W następnej kolejności ustawia autostart server NFS oraz go uruchamia.\\
W drugiej kolejności, następuje instalacja \textit{git}-a.
Ostatnią wykonywaną rzeczą, jest \textit{deploy} wszystkich aplikacji.
\textit{Deploy} wykonywany jest do aktualnej wersji w gałęzi \textit{master}.
\subsection{Director}
Do skonfigurowania \textit{Directora}, potrzebna jest instalacja pakietu \texttt{ipvsadm}, który dostarcza narzędzia do konfiguracji \textit{Linux Virtual Server}.
Konfiguracja \textit{LVS} przeprowadzana jest poprzez użycie mechanizmu zapisu i odczytu aktualnej tablicy \textit{LVS}.
Tuż po instalacji, wykonywany jest zapis konfiguracji, w celu przeprowadzenia całej procedury zapisu tablicy do pliku.
Następnie, generowany jest nowy plik konfiguracji na podstawie parametrów zadanych przez użytkownika.
Plik ten jest wgrywany na serwer i podmienia poprzednio utworzony przy poleceniu zapisu.
Następnie wykonywana jest procedura wczytywania tablicy z pliku do aktualnie działającej instancji.
W efekcie, tablica wygenerowana przez system staje się aktualnie działającą.
Następnie ustawia się autouruchamianie usługi \texttt{LVS}.\\
W drugiej kolejności, tworzony jest wirtualny interface sieciowy, oraz zostają mu przypisane adresy IP odpowiednie dla konkretnych projektów.\\
Każdy projekt nasłuchuje na dedykowanym sobie adresie IP.
Daje to możliwość dedykowania konkretnych serwerów WWW dla projektów, zamiast przypisywać obsługę wszystkich serwerów dla każdego projektu.
\subsection{Real server}
Konfiguracja \textit{real server}-ów jest zbliżona do \textit{director}-a.
Następuje stworzenie wirtualnego interface-u a następnie przypisanie mu odpowiednich adresów IP.
Ważną różnicą w przypadku \textit{real server}-ów jest zapewnienie, aby \textit{real server}-y nie odpowiadały na zapytania \texttt{ARP}.
Uzyskiwane jest to poprzez użycie \texttt{arptables}.
System blokuje wszystkie pakiety typu \textit{ARP response} i pochodzące z adresacji używanej przez \textit{LVS} do nasłuchiwania przez projekty.
\subsection{Server WWW}

\addcontentsline{toc}{chapter}{Podsumowanie}
\chapter*{Podsumowanie}
Niniejsza praca wykazała potrzebę stosowania klastrów WWW.
Przy obecnym rozwoju internetu, ilość zapotrzebowania na dane jest daleko wykraczająca poza możliwości pojedynczych komputerów.
Ponadto, niemożność dostarczenia klientowi żądanych danych jest równoznaczne z ponoszonymi przez firmę stratami finansowymi oraz wizerunkowymi.

Trzeba jednak pamiętać, że nawet najlepszy klaster WWW nie zastąpi optymalizacji aplikacji działającej 

\addcontentsline{toc}{chapter}{Bibliografia}
\bibliography{szz}
\newpage
\addcontentsline{toc}{chapter}{Spis rysunków}
\listoffigures 
\newpage
\addcontentsline{toc}{chapter}{Spis listingów}
\lstlistoflistings
\newpage
\addcontentsline{toc}{chapter}{Spis tabel}
\listoftables
\end{document}
